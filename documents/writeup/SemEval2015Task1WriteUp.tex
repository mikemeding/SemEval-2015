%
% File naaclhlt2013.tex
%

\documentclass[11pt,letterpaper]{article}
\usepackage{naaclhlt2013}
\usepackage{times}
\usepackage{latexsym}
\setlength\titlebox{6.5cm}    % Expanding the titlebox

\title{SemEval 2015 Task 1: Paraphrase and Semantic Similarity in Twitter}

\author{
	Michael Meding\\
  	University Massachusetts Lowell\\
	1 University Ave\\
	Lowell, MA 01854, USA\\
   {\tt mikeymeding@gmail.com}
	\And  
   Hoanh Nguyen\\
	University Massachusetts Lowell\\
	1 University Ave\\
	Lowell, MA 01854, USA\\
	{\tt hoanh.lam.nguyen@gmail.com }
}

\date{}
%	PAPER MUST INCLUDE
% a. an abstract, describing briefly what you have done and results you obtained
%   b. an introduction, a statement of the problem you are trying to address and a brief description of your solution
%   c. related work section, describing relevant results from other people's efforts to solve this problem
%   d. description of your methodology, including 
%       - machine learning methods, 
%       - data sets used in the study,
%       - experimental setup and and evaluation methods;
%   f. description of your results.
%   g. discussion of results, conclusions of your study, future directions for this work

\begin{document}
\maketitle
\begin{abstract}
Hoanh and I decided that we would do the SemEval 2015 Task 1 for our NLP project this semester. This task involves paraphrase and semantic similarity in Twitter which is formalized as follows, Given two sentences, the participants are asked to determine whether they express the same or very similar meaning and optionally a degree score between 0 and 1. Following the literature on paraphrase identification, we evaluate system performance primarily by the F-1 score and Accuracy against human judgements. 
\end{abstract}

\section{Introduction}
\paragraph{} 
Our first task with this project was to translate the original starting code from Python to Java. This required rewriting both the main logistic regression function to a Hidden Markov Model with no hidden layers and to rewrite the data parser so as to interface with the same data that was given to us for this task.

% more explanation of how long this took

\section{Data}
SemEval provides all of its tasks with data for use with evaluating the results of your work. In our prior Twitter project we hand annotated our own data set which was 
extremely time consuming and frustrating. We also did not have any kind of verification of our data set so the accuracy left much to be desired. Luckily, the data sets which were provided to us for this task are both consistent and accurate. Below is an example of the data provided to us.

% insert table here
% | Topic_Id | Topic_Name | Sent_1 | Sent_2 | Label | Sent_1_tag | Sent_2_tag |
% add explanation of the data set and why it is organized that way.

% The "Trending_Topic_Name" are the names of trends provided by Twitter, which are
%  not hashtags.
  
The "Sent\_1" and "Sent\_2" are the two sentences, which are not necessarily full 
tweets. Tweets were tokenized (thanks to Brendan O'Connor et al.) and 
split into sentences. 
 
The "Label" column is in a format such like "(1, 4)", which means among 5 votes 
from Amazon Mechanical turkers only 1 is positive and 4 are negative. We would 
suggest map them to binary labels as follows:
paraphrases: (3, 2) (4, 1) (5, 0)
non-paraphrases: (1, 4) (0, 5)
debatable: (2, 3)  which you may discard if training binary classifier
The "Sent\_1\_tag" and "Sent\_2\_tag" are the two sentences with part-of-speech 
and named entity tags (thanks to Alan Ritter). 

\section{Base Line}
The baseline implementation used a logistic regression model and simple lexical features. The features made uses of unigrams, bigrams, and trigrams of the words and the porter stem of the words. It calculates the precision (intersection verses original ngrams), recall (intersection verses candidate ngrams), and the F1 providing a total of 9 features. Our reimplementation of the baseline was able to achieve an F1 score of just over 0.5 on the dev set after training on the training set. This is only slightly worse than the implementation that was provided. The results from the given Python baseline was 0.5 for the F1 score which with later tweeks we managed to get very close to.

% baseline results table

\section{Data}
% data organization and 
The data which was provided to us was consisted of two files, a training data set and a development data set. The training data set consisted of 13063 Tweet pairs and the development data set consisted of 4727 Tweet pairs. These data sets were organized as tab separated values categorized as such,

% insert table showing categories and examples
% | Topic_Id | Topic_Name | Sent_1 | Sent_2 | Label | Sent_1_tag | Sent_2_tag |
 
% Talk about what the trending topic names and how it relates to our

\section{Base Line Modification}

%What trend? What is the trend and how does it look?

Our first thought was to see what we could do to improve the baseline. After some time and research we decided to see what would happen if we simply omitted the trend. So now the features would make uses of ngrams before and after the trend. With that change the F1 score on the dev set went up to over 0.54. Seeing this we elected to have all our features use the same approach of omitting the trend.

\section{Ark Tweet NLP}
Ark Tweet NLP is a part-of-speech tagger that was built specifically for Tweets. In a previous project we created features that looked at ngrams of the tags and those features worked quite well. For this task we followed the idea introduced in the original python baseline and calculated the precision, recall, and F1. The actual F1 score on the dev set was just shy of 0.36. % was this for our baseline or theirs?

\section{Harvard General Inquirer}
The Harvard General Inquirer is a lexical resource that provides a number of categories that a word belongs to. There are 182 categories however most don't show up very often. For our data set the Harvard Inquirer categories that we settled on were those that appeared more than 3000 times in the training set. For features we took a bag of words approach and used the categories found in the original and candidate tweets. Aside from that we again calculated the precision, recall, and F1 of the mutual category count verses the category count for the original and candidate tweets. The F1 score of these features was just under 0.31.

\section{SentiWordNet}
SentiWordNet is a lexical resource that provides sentiment scores for words. Each SentiWordNet entry contains five explicit attributes (part-of-speech, id, positive score, negative score, synset terms, and glossary) and an implicit attribute (objective which is 1 - positive score - negative score). 
% table here showing what sentiwordnet looks like
The intuition behind using sentiment is if one statement has a positive sentiment and the other has a negative sentiment then it would likely not be a paraphrase. The features we implemented were scores divided by entry count, if there are more negative entries than positive entries, scores divided by non-zero score entry count, scores for adjectives divided by non-zero score entry count, and binary features for majority counts. These features looked at each statement individually. Most of these features were inspired by the Opinion Mining Using SentiWordNet paper. Some features that looked at both statements were binary features that check if both had majority score or counts. The F1 score of these features was around 0.06.

%Quantifiy results of sentiword net alone... Need table

Our first experiment after establishing a good baseline was to score the tweets based on word level sentiment. We preformed a crude run with sentiment weighted heavily to see if we could get any kind of result. This unfortunately was quite bad and did not yield an improvement of any kind. We pushed a bit further by attempting to score the entire tweet and getting an average for comparison but the results were equally as bad. During this time we had acquired Willie Boag's Twitter sentiment code from a prior SemEval competition to see if his sentiment analyser could improve on our crude model. However, after seeing the dismal results using only a crude model we decided that it would be of more value to pursue other features to improve our model.

\section{MPQA Subjective Lexicon}
The MPQA Subjectivity Lexicon contains entries and provide the strength of the subjectivity and polarity. For this resource we created a number of binary features. The features compare the number of negative polarity counts to positive polarity counts and another was designed to compare the total number of weak counts to strong counts and negative counts to positive counts. The F1 score of these features was about 0.13.


\section{Wordnet Synonym}

%Need an inital explanation...

\section{Harvard Inquirer with Wordnet Synonym}

\section{Final Results}
% table of our results by comparision to others
% This is quite a large table...


\section*{Acknowledgments}



% be sure to document all references saved for this project in the references folder

%\begin{thebibliography}{}

%\bibitem[\protect\citename{Aho and Ullman}1972]{Aho:72}
%Alfred~V. Aho and Jeffrey~D. Ullman.
%\newblock 1972.
%\newblock {\em The Theory of Parsing, Translation and Compiling}, volume~1.
%\newblock Prentice-{Hall}, Englewood Cliffs, NJ.
%
%\bibitem[\protect\citename{{American Psychological Association}}1983]{APA:83}
%{American Psychological Association}.
%\newblock 1983.
%\newblock {\em Publications Manual}.
%\newblock American Psychological Association, Washington, DC.
%
%\bibitem[\protect\citename{{Association for Computing Machinery}}1983]{ACM:83}
%{Association for Computing Machinery}.
%\newblock 1983.
%\newblock {\em Computing Reviews}, 24(11):503--512.
%
%\bibitem[\protect\citename{Chandra \bgroup et al.\egroup }1981]{Chandra:81}
%Ashok~K. Chandra, Dexter~C. Kozen, and Larry~J. Stockmeyer.
%\newblock 1981.
%\newblock Alternation.
%\newblock {\em Journal of the Association for Computing Machinery},
%  28(1):114--133.
%
%\bibitem[\protect\citename{Gusfield}1997]{Gusfield:97}
%Dan Gusfield.
%\newblock 1997.
%\newblock {\em Algorithms on Strings, Trees and Sequences}.
%\newblock Cambridge University Press, Cambridge, UK.

%\end{thebibliography}

\end{document}
