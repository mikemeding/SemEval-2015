\documentclass[dvips,12pt]{article}

% Any percent sign marks a comment to the end of the line

% Every latex document starts with a documentclass declaration like this
% The option dvips allows for graphics, 12pt is the font size, and article
%   is the style

\usepackage[pdftex]{graphicx}
\usepackage{url}

% These are additional packages for "pdflatex", graphics, and to include
% hyperlinks inside a document.

\setlength{\oddsidemargin}{0.25in}
\setlength{\textwidth}{6.5in}
\setlength{\topmargin}{0in}
\setlength{\textheight}{8.5in}

% These force using more of the margins that is the default style

\begin{document}

% Everything after this becomes content
% Replace the text between curly brackets with your own

\title{SemEval 2015 Task 1}
\author{Mike Meding \& Hoanh Nguyen}
\date{\today}

% You can leave out "date" and it will be added automatically for today
% You can change the "\today" date to any text you like


\maketitle

%By the time you submit your project proposal, you should know what system you will be implementing as your baseline.
%
%A project proposal should contain:
%− A statement of the problem: problem definition which clearly specifies input and output.
%− A review of the relevant literature; identify the state-of-the-art approach you will implement as your baseline.
%− An outline of the approach you plan to take: 
%   (a) A description of the algorithms and approaches you plan to implement 
%   (b) Tools you are planning to use
%− An overall plan for experiments & evaluation
%− Real world data-set(s) that will be used for evaluation
%
%Project proposals are due Mon Mar 2. 
%
%Plese submit your project proposals as follows:
%
%   $ submit arum project-proposal items-to-submit


\section{Introduction}

%− A statement of the problem: problem definition which clearly specifies input and output.
\paragraph{}
Our choice for the project this semester is SemEval 2015 Task 1. This task has to do with paraphrasing tweets to divulge similarity between them. In bold below is the problem as stated directly from the SemEval website.
\textbf{Given two sentences, the participants are asked to determine whether they express the same or very similar meaning and optionally a degree score between 0 and 1.}


\section{Relevant Text}
%− A review of the relevant literature; identify the state-of-the-art approach you will implement as your baseline.


\section{Approach}
%− An outline of the approach you plan to take: 
%   (a) A description of the algorithms and approaches you plan to implement 
\paragraph{}
The great part about doing a project from SemEval is that all baseline models and data are given. For our baseline we have the option of choosing between two models. One is a logistic regression model with an f-score of 0.6, the other model is a weighted factorization matrix. \cite{gonzalez2012}

%   (b) Tools you are planning to use
\paragraph{}
Given these baseline models we will likely improve them using state-of-the-art semantic analysis tools such as word2vec. Additi

\section{Experiments \& Evaluation}
%− An overall plan for experiments & evaluation

\section{Data}
%− Real world data-set(s) that will be used for evaluation
 
 
\begin{thebibliography}{99}

\bibitem{gonzalez2012} Jonay I. Gonz\'{a}lez Hern\'{a}ndez, 
Pilar Ruiz-Lapuente,	
Hugo M. Tabernero,	
David Montes,	
Ramon Canal,	
Javier M\'{e}ndez	
and Luigi R. Bedin,
{No surviving evolved companions of the progenitor of SN1006},
Nature, {\bf 489}, 533-536 (2012).

\end{thebibliography}



\end{document}
